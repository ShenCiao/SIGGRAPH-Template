\begin{figure}[htb]
  \centering
  \subfloat[raster]{
    \label{fig:example:raster}
    \includegraphics[width=0.4\linewidth]{\CurrentFilePath/161.jpg}
  }%subfloat
  \subfloat[vector]{
    \label{fig:example:pdf}
    \includegraphics[width=0.58\linewidth]{\CurrentFilePath/example.pdf}
  }%subfloat
  
  \subfloat[vector + raster]{
    \label{fig:example:svg_img}
    \includegraphics[width=0.9\linewidth]{\CurrentFilePath/svg_img_example.pdf}
  }%subfloat
 \Caption{Example figure.}
 {%
\subref{fig:example:raster} is a raster image and \subref{fig:example:pdf} is a vector graphics.
Never, ever, rasterize vector graphics unless you want large size and low quality files.
We can also combine vector and raster graphics as in \subref{fig:example:svg_img}.
Use Inkscape to import an image into a pdf file, and draw over it using whatever stuff like texts or strokes.
\note{Particularly regarding the text, always export images to PDF to keep the text selectable, like 'hello' in the figure.}

Command "includegraphics" and "input" always use the root directory as working directory.
So, use the command "CurrentFilePath" rather than "./" if we need relative path.
 }
 \label{fig:example}
\end{figure}